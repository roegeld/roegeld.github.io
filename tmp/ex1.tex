\documentclass[a4paper,12pt]{article}
\usepackage[utf8]{inputenc}
\usepackage[T1]{fontenc}
\usepackage[french]{babel}
\usepackage{palatino}

\begin{document}
\title{Révisions entiers signés et réels}
\author{}
\date{}
\maketitle

\section{Entiers}

\begin{enumerate}
\item Soient les quatre nombres hexadécimaux codés sur 8 bits suivants :
$(46)_{16}$ , $(C6)_{16}$ , $(24)_{16}$ , $(CB)_{16}$
\begin{enumerate}
\item Convertir ces nombres en décimal en considérant les deux cas :
  a)~non signés, et b)~signés.
\item Convertir ces nombres sur 16 bits en considérant les cas précités
\end{enumerate}

\end{enumerate}


\section{Réels}

\subsection{Conversions décimal $\rightarrow$ binaire}

\begin{enumerate}
\item Coder les réels suivants selon la norme IEEE 754 32 bits : 8, 9, 1.5, 3.14, $-6.625$, et 125.
\item En virgule flottante normalisée, coder en binaire au format simple précision le réel 12.575 , puis
  effectuer le codage inverse.
\item Convertir en binaire 0.5, 0.2, 0.9, $1/3$, $3/8$
\item Convertir en IEEE754 simple précision
  $-52.234375$, $7112.0$, $16.2$~;
  vérifier les conversions en retransformant en nombres décimaux.
  Évaluer les erreurs commises, le cas échéant.

\item Conversion en IEEE 754 16 bits~: il existe aussi le codage
  en «~demi-précision~», c'est un codage des réels analogue
  à celui sur 32 bits et 64 bits, mais seulement sur 16 bits~;
  l'exposant est codé sur 5 bits par excédent à 15~;
  reprendre les exercices précédents pour la demi-précision.

\item Conversion en IEEE 754 (fictif) 8 bits~: on suppose que le premier
  bit est celui du signe, que les trois bits suivants représentent l'exposant
  par excès à 3, et que les quatre derniers bits correspondent à la
  mantisse normalisée (sans le 1 initial).
  On suppose que les exposants 000 et 111 sont réservés.
  Quel est alors le plus petit nombre positif représentable~? Et le plus grand~?

  
 
  
\end{enumerate}

\subsection{Conversions binaire $\rightarrow$ décimal}

\begin{enumerate}
\item En virgule fixe, décoder le nombre binaire 11.011

\item Convertir en décimal, les nombres hexadécimaux réels données
  sous format IEEE 754 32 bits~:
42E48000, 3F880000 , C7F00000, BFC00000, C0900000, 80000008.

\item Étant donnés les nombres $( 0.10010\cdot 10101 )_2$,
et $( 0.11010 \cdot 101 )_2$, effectuer leurs somme et
produit en virgule flottante

\item Allez sur la page \verb|https://www.browserling.com/tools/random-hex|
  et générer 10 nombres hexadécimaux à 8 chiffres. Convertissez ces
  nombres en décimal, en supposant qu'ils sont codés en IEEE754.
  Vous pouvez vous aider du programme \texttt{bc}.
  Vérifiez ensuite vos conversions en faisant les
  opérations inverses.

\item Faites de même, mais avec 16 chiffres (64 bits).
\end{enumerate}

\end{document}
